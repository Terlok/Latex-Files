\documentclass[a4paper,12pt]{article}
\usepackage[english,ukrainian,russian]{babel}
\linespread{1.5}
\usepackage{ucs}
\usepackage[utf8]{inputenc}
\usepackage[T2A]{fontenc}
\usepackage{amsmath}
\usepackage{bigints}
\usepackage{amsfonts}
\usepackage{graphicx}
\usepackage{amssymb}
\usepackage{cancel}
\usepackage{gensymb}
\usepackage[pageanchor]{hyperref}
\newcommand{\dx}{\textbf{d}x}
\newcommand{\dt}{\textbf{d}t}
\newcommand{\du}{\textbf{d}u}
\newcommand{\dv}{\textbf{d}v}
\newcommand{\dy}{\textbf{d}y}
\newcommand{\dz}{\textbf{d}z}
\newcommand{\arch}{\textrm{arcch}}
\newcommand{\arsh}{\textrm{arcsh}}
\newcommand{\dint}{\displaystyle\int}
\newcommand\tab[1][1cm]{\hspace*{#1}}
\newcommand{\dsum}{\displaystyle\sum}
\usepackage[left=20mm, top=20mm, right=15mm, bottom=15mm, nohead, nofoot]{geometry}



\begin{document}
	\begin{center}
		\vspace*{0,1cm}
		{\Large \bfseries \textsc{Домашнє завдання з фізики\\
				Студента групи ФІ-12 Завалій Олександра}}\\
		\hrulefill
	\end{center}
	\begin{center}
		\textbf{Завдання №1}
	\end{center}
	$
	\vec{c}=x\vec{i}+2z\vec{k} \Rightarrow \vec{c}=(x, 0, 2z)\\
	\dfrac{\dx}{x}=\dfrac{\dy}{0}=\dfrac{\dz}{2z}\\
	\begin{cases}
		\dfrac{1}{x}\dx = \dfrac{1}{2z}\dz\\
		\dz = 0
	\end{cases}\\
	\dint\dfrac{1}{x}\dx = \dint\dfrac{1}{2z}\dz \Rightarrow \ln(|x|)=\dfrac{1}{2}\ln(|z|) + C \: \bigg| \cdot 2\\
	2\ln(|x|)=\ln(|z|) + C\\
	2\ln(|x|)-\ln(|z|) = C\\
	y=x^2e^c
	$
	
	
	\begin{center}
		\textbf{Завдання №2}
	\end{center}
	$\vec{c}=(3x+yz)\vec{i}+(3y+xz)\vec{j}+(3z+xy)\vec{k} \Rightarrow \vec{c}=(3x+yz,\: 3y+xz,\: 3z+xy) \\
	rot\vec{c}=
	\begin{bmatrix} 
		\vec{i} & \vec{j} & \vec{k} \\
		\dfrac{\partial}{\partial x} & \dfrac{\partial}{\partial y} & \dfrac{\partial}{\partial z}\\
		3x+yz & 3y+xz & 3z+xy \\
	\end{bmatrix}
	=\bigg(\dfrac{\partial}{\partial y}(3z+xy) - \dfrac{\partial}{\partial z}(3y+xz)\bigg)\vec{i}-\\
	-\bigg(\dfrac{\partial}{\partial x}(3z+xy) - \dfrac{\partial}{\partial z}(3x+yz)\bigg)\vec{j}+
	\bigg(\dfrac{\partial}{\partial x}(3y+xz) - \dfrac{\partial}{\partial y}(3x+yz)\bigg)\vec{k}=\\
	=(x-x)\vec{i}-(y-y)\vec{j}+(z-z)\vec{k}=0\cdot\vec{i}-0\cdot\vec{j}+0\cdot\vec{k}=0$ \\
	$div\vec{c} = \dfrac{\partial}{\partial x}(3x+yz) + \dfrac{\partial}{\partial y}(3y+xz) + \dfrac{\partial}{\partial z}(3z+xy)=3+3+3=9$\\
	\\Оскільки $rot\vec{c}=0$ і $div\vec{c} \neq 0 \Rightarrow $ це поле потенціальне.
	
	
	\newpage
	\begin{center}
		\textbf{Завдання №3}
	\end{center}
	$\vec{c}= (xy-2x)\vec{i}+(xz+2y)\vec{j}+xy\vec{k} \Rightarrow \vec{c}=(xy-2x,\: xz+2y,\: xy)\\$
	$
	rot\vec{c}=
	\begin{bmatrix} 
		\vec{i} & \vec{j} & \vec{k} \\
		\dfrac{\partial}{\partial x} & \dfrac{\partial}{\partial y} & \dfrac{\partial}{\partial z}\\
		xy-2x & xz+2y & xy \\
	\end{bmatrix}=
	\bigg(\dfrac{\partial}{\partial y}(xy) - \dfrac{\partial}{\partial z}(xz+2y)\bigg)\vec{i}-\\
	-\bigg(\dfrac{\partial}{\partial x}(xy) - \dfrac{\partial}{\partial z}(xy-2x)\bigg)\vec{j}+
	\bigg(\dfrac{\partial}{\partial x}(xz+2y) - \dfrac{\partial}{\partial y}(xy-2x)\bigg)\vec{k}=\\
	=(x-x)\vec{i}-(y-0)\vec{j}+(z-x)\vec{k}=0\vec{i}-(y)\vec{j}+(z-x)\vec{k}\neq0$\\
	$div\vec{c}= \dfrac{\partial}{\partial x}(xy-2x) + \dfrac{\partial}{\partial y}(xz+2y) + \dfrac{\partial}{\partial z}(xy)=y-2+2+0=y$\\
	\\Оскільки 
	$\begin{cases}
		rot\vec{c}\neq0\\
		div\vec{c} \neq 0
	\end{cases}
	\Rightarrow $ це поле не вихрове, не потенціальне та не гармонічне.
	
	
	
	
	
	
	
	
	
	
	
\end{document}








