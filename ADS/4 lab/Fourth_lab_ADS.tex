\documentclass[a4paper,12pt]{article}
\usepackage[english,ukrainian,russian]{babel}
\linespread{1}
\usepackage{ucs}
\usepackage[utf8]{inputenc}
\usepackage[T2A]{fontenc}
\usepackage[paper=portrait,pagesize]{typearea}
\usepackage{amsmath}
\usepackage{bigints}
\usepackage{amsfonts}
\usepackage{graphicx}
\usepackage{amssymb}
\usepackage{cancel}
\usepackage{gensymb}
\usepackage{multirow}
\usepackage{rotate} 
\usepackage{pdflscape}
\usepackage{bigstrut}
\usepackage[pageanchor]{hyperref}
\usepackage{chngpage}
\usepackage{booktabs}
\newcommand\tab[1][1cm]{\hspace*{#1}}
\newcommand{\RomanNumeralCaps}[1]{\MakeUppercase{\romannumeral #1}}
\usepackage[left=20mm, top=20mm, right=15mm, bottom=15mm, nohead, nofoot]{geometry}

\usepackage{listings}
\usepackage{xcolor}

\definecolor{codegreen}{rgb}{0,0.6,0}
\definecolor{codegray}{rgb}{0.5,0.5,0.5}
\definecolor{codepurple}{rgb}{0.58,0,0.82}
\definecolor{backcolour}{rgb}{0.95,0.95,0.92}

\lstdefinestyle{mystyle}{
	backgroundcolor=\color{backcolour},   
	commentstyle=\color{codegreen},
	keywordstyle=\color{blue},
	numberstyle=\tiny\color{codegray},
	stringstyle=\color{red},
	basicstyle=\ttfamily\footnotesize,
	breakatwhitespace=false,         
	breaklines=true,                 
	captionpos=b,                    
	keepspaces=true,                 
	numbers=none,                    
	numbersep=5pt,                  
	showspaces=false,                
	showstringspaces=false,
	showtabs=false,                  
	tabsize=4,
	frame=shadowbox
}

\lstset{style=mystyle}

%\begin{lstlisting}[language=C++]
%\end{lstlisting}

\begin{document}
    \begin{center}
        \hfill \break
        \large{\textbf{НАЦIОНАЛЬНИЙ ТЕХНIЧНИЙ УНIВЕРСИТЕТ УКРАЇНИ\\
                «КИЇВСЬКИЙ ПОЛIТЕХНIЧНИЙ IНСТИТУТ»\\
                НАВЧАЛЬНО-НАУКОВИЙ ФІЗИКО-ТЕХНІЧНИЙ ІНСТИТУТ}}\\
        \hfill \break \hfill \break \hfill\break \hfill \break \hfill \break \hfill \break \hfill \break
        \hfill \break \hfill \break \hfill \break \hfill \break
        \large{Лабораторна робота № 4}
        \begin{center}
            \normalsize{\textbf{з дисципліни «Алгоритми та структури даних» \\
            На тему: «Структури даних: стеки, черги» \\}}
        \end{center}
    \end{center}
    \hfill \break \hfill \break \hfill \break \hfill \break \hfill \break \hfill \break \hfill \break
    \hfill \break \hfill \break \hfill \break \hfill \break \hfill \break \hfill \break 
    \begin{flushright}
        \large{ \hspace{35pt} Виконав:\\
            студент групи ФI-12\\
            Завалій Олександр} 
    \end{flushright}
    \hfill \break \hfill \break \hfill \break \hfill \break \hfill \break \hfill \break \hfill \break
    \hfill \break
    \begin{center} \textbf{Київ-2023} \end{center}
    \thispagestyle{empty}

\newpage
    \begin{center}
        \Large{\bfseries{Реалізація завдання}} \\
        \Large{Варіант №5}
    \end{center}
    Реалізувати чергу двома способами:
    \begin{enumerate}
        \item Масивом, використавши кільцеву чергу.
        \item Зв'язним списком.
    \end{enumerate}
    \begin{center}
        \Large{Task \RomanNumeralCaps{1}}
    \end{center}
    \textbf{Масив, кільцева черга.}
    \begin{lstlisting}[language=C++]
#include <iostream>
using namespace std;
#define SIZE 6

template <typename T>
class QueueRing {
private:
    T arr[SIZE];
    int front, rear;
public:
    QueueRing() {
        front = rear = -1;
    }

    bool isEmpty();
    bool isFull();
    void enqueue(int num);
    void dequeue();
    void print();
};
template <typename T>
bool QueueRing<T>::isEmpty() {
    return front == -1 and rear == -1;
}

template <typename T>
bool QueueRing<T>::isFull() {
    return (rear + 1) % SIZE == front;
}

template <typename T>
void QueueRing<T>::enqueue(int value) {
    if (isFull()) {
        cout << "Queue is full\n";
        return;
    } else if (isEmpty()) {
        front = rear = 0;
    } else {
        rear = (rear + 1) % SIZE;
    }
    arr[rear] = value;
}
    \end{lstlisting} 

\newpage
    \begin{lstlisting}[language=C++]
template <typename T>
void QueueRing<T>::dequeue() {
    if (isEmpty()) {
        cout << "Queue is empty\n";
        return;
    } else if (front == rear) {
        front = rear = -1;
    } else {
        front = (front + 1) % SIZE;
    }
}

template <typename T>
void QueueRing<T>::print() {
    if (isEmpty()) {
        cout << "Queue is empty\n";
        return;
    }
    cout << "Queue: ";
    for (int i = front; i != rear; i = (i + 1) % SIZE) {
        cout << arr[i] << ' ';
    }
    cout << arr[rear] << endl;
}


int main() {
    QueueRing<char> q;

    for (int i = 97; i < SIZE+97; i++) {
        q.enqueue(i);
    }
    q.print();
    
    q.enqueue(102);

    for (int i = 0; i < 4; i++) {
        q.dequeue();
    }
    q.print();

    q.enqueue(103);
    q.enqueue(104);
    q.enqueue(105);
    q.print();

    return 0;
}        
    \end{lstlisting} 

\newpage
    \begin{center}
        \Large{Task \RomanNumeralCaps{2}}
    \end{center}
    \textbf{Зв'язний список.}
    \begin{lstlisting}[language=C++]
#include <iostream>
#include <string>
using namespace std;

template <typename T>
class Queue {
private:
    struct Node {
        T data;
        Node* next;
    };
    Node* front;
    Node* rear;
public:
    Queue() {
        front = rear = NULL;
    }
    ~Queue() {
        Node* current = front;
        while (current) {
            Node* temp = current->next;
            delete current;
            current = temp;
        }
    }

    void enqueue(T value);
    void dequeue();
    void dequeueA();
    void print();
};
template <typename T>
void Queue<T>::enqueue(T value) {
    Node* tempnode = new Node();
    tempnode->data = value;
    tempnode->next = NULL;
    if (front == NULL and rear == NULL){
        front = rear = tempnode;
    }
    rear->next = tempnode;
    rear = tempnode;
}
template <typename T>
void Queue<T>::dequeue(){
    Node* tempnode = front;
    if (!tempnode) {
        cout << "Queue is empty!" << endl;
        return;
    }
    if (tempnode == rear) {
        front = rear = NULL;
    } else {
        front = tempnode->next;
    }
    delete tempnode;
}
    \end{lstlisting} 

\newpage
    \begin{lstlisting}[language=C++]
template <typename T>
void Queue<T>::dequeueA(){
    Node* tempnode = front;
    while (tempnode) {
        Node* temp = tempnode->next;
        delete tempnode;
        tempnode = temp;
    }
    front = rear = NULL;
}
template <typename T>
void Queue<T>::print(){
    Node* tempnode = front;
    if (!tempnode) {
        cout << "Queue is empty!" << endl;
        return;
    }
    while (tempnode) {
        cout << tempnode->data << ' ';
        tempnode = tempnode->next;
    }
    cout << endl;
}

int main() {
    Queue<string> qlist;
    string temp;
    
    do {
        cin >> temp;
        if (temp != "q"){
            qlist.enqueue(temp);
        } else {
            break;
        }
    } while(true);

    qlist.print();
    qlist.dequeue();
    qlist.print();

    qlist.enqueue("END");
    qlist.print();

    qlist.dequeueA();
    qlist.print();

    return 0;
}        
    \end{lstlisting} 

\end{document}