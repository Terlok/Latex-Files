% !TeX program = lualatex
% !TeX encoding = utf8
% !TeX spellcheck = uk_UA
% !BIB program = biber

\documentclass[a4paper]{article}
\usepackage[fontsize=14pt]{fontsize}
\usepackage{csquotes}
\usepackage{fontspec}
\usepackage[pageanchor]{hyperref}
\usepackage[english,ukrainian]{babel}
\setsansfont{CMU Sans Serif}
\setmainfont{CMU Serif}
\setmonofont{CMU Typewriter Text}
\usepackage[footskip=1cm, headsep=0.3cm, left=2cm, top=2cm, right=2cm, bottom=2cm]{geometry}
\usepackage[backend=biber,style=numeric]{biblatex}
\addbibresource{references.bib}


\begin{document}
    \pagestyle{empty}
    \noindent
    З огляду обраної теми моєї дипломної роботи можна виділити три основні категорії, по яким розподілялись книжки.
    \begin{enumerate}
        \item Вітрові електростанції: \cite{Okedu2023, Nielsen2024, MetaheuristicAlgorithmsinEnergy}
        \item Метаевристики: \cite{Wang2024, Okwu2021}
        \item Додаткова література: \cite{ExamplesEnergyModels, ComputationalIntelligence, Lobato2017, Kaul2021, SS2021, Kumar2020}
    \end{enumerate}
    
    \nocite{*}
    \printbibliography

\end{document}